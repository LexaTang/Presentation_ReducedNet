% !TeX document-id = {c2f20315-5529-42f2-bcf9-afcfdc5d6349}
% !TeX spellcheck = zh_CN
% !TeX encoding = UTF-8
% !TeX TXS-program:compile = txs:///pdflatex/
% !BIB program = biber



\documentclass[UTF8, fontset=founder, aspectratio=43, 10pt, t]{ctexbeamer}

\input{Tex/Preamble_Presentation}

\begin{document}
	
%%%%%%%%%%%%%%%%%%%%%%%%%%%%%%%%%%%%%%%
%%  MISCELLANEOUS SETTINGS/COMMANDS  %%
%%%%%%%%%%%%%%%%%%%%%%%%%%%%%%%%%%%%%%%


\newcommand{\balA}[1][1]{BAL$^\mathup{I}_{#1:#1}$\xspace}
\newcommand{\unbalA}[1][n]{UNBAL$^\mathup{I}_{1:#1}$\xspace}
\newcommand{\balB}[1][1]{BAL$^\mathup{II}_{#1:#1}$\xspace}
\newcommand{\unbalB}[1][n]{UNBAL$^\mathup{II}_{#1:1}$\xspace}

%% Add separator slide to the beginning of each section ==>
\AtBeginSection[]{
	\begin{frame}[standout, c]{~}
		%\vfill
		\usebeamerfont{title}%
		\textcolor{SpotColor}{\insertsectionhead}
		%\vfill
	\end{frame}%
}

%% Alternativelyy, add ``Outline'' slide to the beginning of each section ==>
%\AtBeginSection[]{
%	\begin{frame}[plain]{Outline}
%		\tableofcontents[currentsection]
%	\end{frame}
%}
%% <==

\title{精简卷积神经网络简介}

\subtitle{以MobileNets和SkipNet为例}

\author[唐呈俊]{%
	唐呈俊
} % Author(s)

\institute{%
	桂林电子科技大学
} % Institution(s)

\date{%
	%\\
	%Name of the Inviting %Institution/Seminar Series
	\\[\medskipamount]
	\textmd{\today}%
}




%%%%%%%%%%%%%%%%%%%
%%  TITLE SLIDE  %%
%%%%%%%%%%%%%%%%%%%


\begin{frame}[standout]{~}
	
	\titlepage%
	
\end{frame}


\begin{frame}[standout]{大纲}
	
	\medskip
	\tableofcontents
	
\end{frame}




%%%%%%%%%%%%%%%%%
%%  MAIN PART  %%
%%%%%%%%%%%%%%%%%


\section{卷积}

\begin{frame}{\titleprefix: 卷积层的计算}
	\begin{figure}
		\centering
		\includegraphics[width=\linewidth]{Images/conv_1}
		\caption{卷积计算过程}
		\label{fig:conv1}
	\end{figure}
	
\end{frame}

\begin{frame}{\titleprefix: 卷积层的参数}
	卷积计算的公式:
	\[
	\mathbf{G}_{k, l, n}=\sum_{i, j, m} \mathbf{K}_{i, j, m, n} \cdot \mathbf{F}_{k+i-1, l+j-1, m}
	\]
	卷积层的参数:
	\begin{description}
		\item[Filter] 滤波器数量
		\item[Kernel Size] 卷积核大小
		\item[Strides] 步长
		\item[Padding] 填充
	\end{description}
	
\end{frame}

\section{MobileNets}

\begin{frame}{\titleprefix: AlexNet}
	虽然AlexNet并非轻量化神经网络,但其中的两个方法,却是如今轻量化网络的重要基石:
	\begin{description}
		\item[Group Conv] AlexNet中,为了在两个GPU上并行训练,其将卷积按照通道分解为两个Group,从而在两个GPU上完成。
		\item[ReLU] AlexNet中,采用了ReLU激活函数,相比先前的tanh和sigmoid函数,ReLU函数的计算速度,和训练时的收敛速度都更快。
	\end{description}
	\[
	\text{ReLU}(x)=\max (0, x)
	\]
	
\end{frame}

\begin{frame}{\titleprefix: Group Conv}
	\begin{figure}
		\centering
		\includegraphics[width=0.9\linewidth]{Images/groupconv_1}
		\caption{AlexNet的Group Conv结构}
		\label{fig:groupconv1}
	\end{figure}

\end{frame}

\begin{frame}{\titleprefix: Depthwise Conv}
	虽然AlexNet的Group Conv的目的是将计算分配给两个GPU并行运行,但是这也同样减少了计算量。
	
	如果说将每个通道,都分到其对应的Group中,这种Group Conv的特殊情况,就将其运算量减少到了最少的情况。但这样做的代价是通道之间的互信息被忽略了。
	\begin{figure}
		\centering
		\includegraphics[width=0.7\linewidth]{Images/depthwise_1}
		\caption{Depthwise Conv,其中M是通道数,也是滤波器数}
		\label{fig:depthwise1}
	\end{figure}
	
\end{frame}

\begin{frame}{\titleprefix: Depthwise Conv}
	\begin{figure}
		\centering
		\includegraphics[width=0.5\linewidth]{Images/depthwise_2}
		\caption{标准的Conv(相对于 Depthwise Conv)}
		\label{fig:depthwise2}
	\end{figure}
	\[
	\mathbf{G}_{k, l, n}=\sum_{i, j, m} \mathbf{K}_{i, j, m, n} \cdot \mathbf{F}_{k+i-1, l+j-1, m}
	\text{普通卷积}
	\]	
	\[
	\hat{\mathbf{G}}_{k, l, m}=\sum_{i, j} \hat{\mathbf{K}}_{i, j, m} \cdot \mathbf{F}_{k+i-1, l+j-1, m}
	\text{Depthwise Conv}
	\]

\end{frame}

\begin{frame}{\titleprefix: MobileNets}
	\begin{figure}
		\centering
		\includegraphics[width=0.5\linewidth]{Images/pointwise_1}
		\caption{Pointwise Conv,即$1\time 1$ 结构的普通卷积}
		\label{fig:pointwise1}
	\end{figure}
	
	计算复杂度:
	\[
	D_{K} \cdot D_{K} \cdot M \cdot N \cdot D_{F} \cdot D_{F} 
	\text{ 普通卷积}
	\]	
	\[
	D_{K} \cdot D_{K} \cdot M \cdot D_{F} \cdot D_{F}+M \cdot N \cdot D_{F} \cdot D_{F} 
	\text{ Depthwise + Pointwise Conv}
	\]
	
	\[
	\frac{D_{K} \cdot D_{K} \cdot M \cdot D_{F} \cdot D_{F}+M \cdot N \cdot D_{F} \cdot D_{F}}{D_{K} \cdot D_{K} \cdot M \cdot N \cdot D_{F} \cdot D_{F}} = \frac{1}{N}+\frac{1}{D_{K}^{2}} 
	\]
\end{frame}

\begin{frame}{\titleprefix: 模型设计}
	\begin{figure}
		\centering
		\includegraphics[width=0.7\linewidth]{Images/mobilenetblock}
		\caption{MobileNets模块}
		\label{fig:mobilenetblock}
	\end{figure}

\end{frame}

\begin{frame}{\titleprefix: 模型设计}
	\begin{figure}
		\centering
		\includegraphics[width=0.5\linewidth]{Images/mobilenet}
		%\caption{MobileNets网络结构}
		\label{fig:mobilenet}
	\end{figure}
\end{frame}

\begin{frame}{\titleprefix: 模型性能}
	\begin{figure}
		\centering
		\includegraphics[width=0.7\linewidth]{Images/mobilenetres}
		%\caption{计算性能消耗}
		\label{fig:mobilenetres}
	\end{figure}

	同时,为了进一步压缩模型大小,MobileNets设计了超参数\alert{宽度乘子}$\alpha$,使得MobileNets模块的输入通道数变为$\alpha M$,输出通道数变为$\alpha N$,此时计算复杂度变为:
	$$
	D_{K} \cdot D_{K} \cdot \alpha M \cdot D_{F} \cdot D_{F}+\alpha M \cdot \alpha N \cdot D_{F} \cdot D_{F}
	$$
	
\end{frame}

\begin{frame}{\titleprefix: 模型性能}
\begin{figure}
	\centering
	\includegraphics[width=0.8\linewidth]{Images/mobilenetpf}
	%\caption{}
	\label{fig:mobilenetpf}
\end{figure}
\end{frame}

\section{SkipNet}

\begin{frame}{Highway Networks}
	\alert{Gating units} 在Highway Networks中,引入了这一概念。如果说原始的神经网络可以表示为:
	\begin{equation}
	\mathbf{y}=H\left(\mathbf{x}, \mathbf{W}_{\mathbf{H}}\right)
	\end{equation}
	引入$T\left(\mathbf{x}, \mathbf{W}_{\mathbf{T}}\right)$, $C\left(\mathbf{x}, \mathbf{W}_{\mathbf{C}}\right)$两个函数,作为变换gate和运载gate:
	\begin{equation}
	\label{tc}
	\mathbf{y}=H\left(\mathbf{x}, \mathbf{W}_{\mathbf{H}}\right) \cdot T\left(\mathbf{x}, \mathbf{W}_{\mathbf{T}}\right)+\mathbf{x} \cdot C\left(\mathbf{x}, \mathbf{W}_{\mathbf{C}}\right)
	\end{equation}
	设$C=1-T$,公式\eqref{tc}变换为:
	\begin{equation}
	\mathbf{y}=H\left(\mathbf{x}, \mathbf{W}_{\mathbf{H}}\right) \cdot T\left(\mathbf{x}, \mathbf{W}_{\mathbf{T}}\right)+\mathbf{x} \cdot\left(1-T\left(\mathbf{x}, \mathbf{W}_{\mathbf{T}}\right)\right)
	\end{equation}
	又令$H'=H\cdot T$:
	\begin{equation}
	\mathbf{y}=H‘\left(\mathbf{x}, \mathbf{W}_{\mathbf{H}}\right)+\mathbf{x} \cdot\left(1-T\left(\mathbf{x}, \mathbf{W}_{\mathbf{T}}\right)\right)
	\end{equation}
	可以看到,加入一个与原网络模块并列的gating unit 即可起到“加速信息流通”的作用。
	
\end{frame}

\begin{frame}{Gating策略}
	实际上,ResNet的结构便符合这一标准:
	\begin{equation}
	\mathbf{x}_{\text {ResNet }}^{i+1}=F^{i}\left(\mathbf{x}_{\text {ResNet }}^{i}\right)+\mathbf{x}_{\text {ResNet }}^{i}
	\end{equation}
	\begin{figure}
		\centering
		\includegraphics[width=0.7\linewidth]{Images/gatingdesign}
		\caption{两种不同的gating策略设计}
		\label{fig:gatingdesign}
	\end{figure}
	作者针对ResNet设计了两种gating策略,一种是所有模块共用一个gate unit,还有一种是对每一个模块施加一个gate unit。
	
\end{frame}

\begin{frame}{Gating units设计}
	针对不同的网络模块类型,作者设计了不同的Gating units:
	\begin{figure}
		\centering
		\includegraphics[width=0.7\linewidth]{Images/gatingdesign1}
		\caption{三种不同的gating units}
		\label{fig:gatingdesign1}
	\end{figure}
	\begin{description}
		\item[FFGate-I] 大约消耗残差模块的$19\%$的性能;
		\item[FFGate-II] 大约消耗残差模块的$12.5\%$的性能;
		\item[RNNGate] 包含了单个LSTM模块,仅消耗$0.04\%$的微不足道的性能。
	\end{description}
	
\end{frame}

\begin{frame}{Hybrid RL}
	对于模块的两个输出大小的估计,直接使用soft-max策略进行训练,随后在计算向后传播时恢复到硬判定使得模型的准确率非常的低,因此作者设计了混合强化学习策略,将跳过模块省略的计算作为“奖励”,因此得到的新的目标函数为:
	\begin{equation}
	\begin{aligned} \min \mathcal{J}(\theta) &=\min \mathbb{E}_{\mathbf{x}} \mathbb{E}_{\mathbf{g}} L_{\theta}(\mathbf{g}, \mathbf{x}) \\ &=\min \mathbb{E}_{\mathbf{x}} \mathbb{E}_{\mathbf{g}}\left[\mathcal{L}\left(\hat{y}\left(\mathbf{x}, F_{\theta}, \mathbf{g}\right), y\right)-\frac{\alpha}{N} \sum_{i=1}^{N} R_{i}\right] \end{aligned}
	\end{equation}
	其中$R_{i}=\left(1-g_{i}\right) C_{i}$作为跳过计算的奖励,常数$C_{i}$为计算消耗,而ResNet中,可以视为$C_{i}=1$。
	
\end{frame}

\begin{frame}{最终的训练策略}
	\begin{figure}
		\centering
		\includegraphics[width=0.7\linewidth]{Images/hrl}
		\caption{最终的训练策略}
		\label{fig:hrl}
	\end{figure}
	$$
	G_{\text {relax }}(\mathbf{x})=\left\{\begin{aligned} \mathbb{I}(S(\mathbf{x}) \geq 0.5), & \text { forward pass } \\ S(\mathbf{x}), & \text { backward pass } \end{aligned}\right.
	$$
\end{frame}

\begin{frame}{模型性能}
	\begin{figure}
		\centering
		\includegraphics[width=0.7\linewidth]{Images/res}
		\caption{模型性能对比}
		\label{fig:res}
	\end{figure}
	\begin{figure}
		\centering
		\includegraphics[width=0.7\linewidth]{Images/difficult}
		\caption{跳过的模块数量与输入的复杂度有关}
		\label{fig:difficult}
	\end{figure}
	
\end{frame}

\section{更多工作}

\begin{frame}{\titleprefix: 网络性能对比实验}
	\begin{figure}
		\centering
		\includegraphics[width=0.7\linewidth]{Images/mobilenetcomp}
		\includegraphics[width=0.7\linewidth]{Images/mobilenetcompgpu}
		\caption{各版本MobileNet在CPU和GPU上性能的表现}
		\label{fig:mobilenetcomp}
	\end{figure}
	
\end{frame}

\begin{frame}{\titleprefix: 如何改进现有网络}
	
	根据今天讲到的网络特性,对于不同的实际使用环境,如何改进现有网络?
	
\end{frame}

\begin{frame}{\titleprefix: MobileNet V2}
	\begin{figure}
		\centering
		\includegraphics[width=0.6\linewidth]{Images/mobilenetv2}
		\caption{MobileNet V2与其它网络的对比}
		\label{fig:mobilenetv2}
	\end{figure}
	
\end{frame}

\begin{frame}{\titleprefix: MobileNet V3}
	\begin{figure}
		\centering
		\includegraphics[width=0.9\linewidth]{Images/mobilenetv3}
		\caption{Sigmoid 和 Swish 的非线性形式及其hard近似}
		\label{fig:mobilenetv3}
	\end{figure}
	\[\text { swish } x=x \cdot \sigma(x)
	\hspace{1in}
	\mathrm{h}-\text { swish }[x]=x
	\frac{\operatorname{Re} \mathrm{L} \mathrm{U} 6(x+3)}{6}\]
\end{frame}

\begin{frame}
	今天的内容结束了,谢谢大家。
\end{frame}

\end{document}